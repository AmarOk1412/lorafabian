\documentclass{article}
%packages
\usepackage{graphicx}
\usepackage{hyperref}
\usepackage[utf8]{inputenc}
\usepackage[T1]{fontenc}
\usepackage[a4paper]{geometry}
\usepackage{minted}

\title{How the lora.begin(name) function works}
\date{July 10th 2015}

\begin{document}
\maketitle

\tableofcontents


\section{Usage}
To change the name of an arduino object, you must use the
\begin{minted}{c}
void LoraShield::begin(String dns);
\end{minted}
function after calling the init() function. For example:
\begin{minted}{c}
#include <LoraShield.h>
#include <SPI.h>

LoraShield lora;
void setup()
{
  lora.init();
  String name = "beta.s.ackl.io";
  lora.begin(name);
}

void loop()
{ }
\end{minted}
\section{Arduino side}
This function is defined in \emph{LoraShield.cpp}:
\begin{minted}{c}
/**
 * \description: send the name of the object to the shield
 * \param: name - the name
 */
void LoraShield::begin(String name)
{
  digitalWrite(SS_PIN, LOW);

  // The byte is the status of the last command
  int previous_cmd_status = SPI.transfer(ARDUINO_CMD_HOSTNAME); 
  delayMicroseconds(WAIT_TIME_BETWEEN_BYTES_SPI);
  
  // Data size to be sent
  // MSB
  int shield_status = SPI.transfer(name.length() >> 8);
  delayMicroseconds(WAIT_TIME_BETWEEN_BYTES_SPI);

  shield_status = SPI.transfer(dns.length());
  delayMicroseconds(WAIT_TIME_BETWEEN_BYTES_SPI);

  //Send:  payload as bytes to send
  for (int i = 0; i < name.length() ; i++)
  {
    shield_status = SPI.transfer(name[i]);
    delayMicroseconds(WAIT_TIME_BETWEEN_BYTES_SPI);
  }

  digitalWrite(SS_PIN,HIGH); 
  delay(WAIT_TIME_BETWEEN_SPI_MSG*2);
}
\end{minted}
So the contiki board will receive a packet like :
\begin{tabular}{|c|c|c|c|}
\hline
0x21 & length\_msb & length\_lsb & name\\
\hline
\end{tabular}
\section{Contiki side}
\subsection{Change the name}
The parsing of the previous packet is defined in the\\ \emph{/platform/lorafabian/apps/arduino\_interface/arduino\_cmd.c} file. The program waits a ARDUINO\_CMD\_HOSTNAME and write the new name in the file: \emph{/HOSTNAME\_LORA}.
\subsection{Get the current name}
The current name is written in the file \emph{/HOSTNAME\_LORA}. This is an example of a function which get the current name (updateHOSTNAME() in the file\\ \emph{/platform/lorafabian/apps/frame\_manager/frame\_manager.c}):
\begin{minted}{c}
#include "cfs/cfs.h"

char coap_payload_beacon[150];

/**
 * \brief: update the hostname with /HOSTNAME file
 */
void updateHOSTNAME()
{
  char dns[150];//The content of the file
  int fd;
  //Read in /HOSTNAME_LORA
  fd = cfs_open("/HOSTNAME_LORA", CFS_READ);
  if(fd >= 0) {
    //Read 500 char
    cfs_read(fd, dns, sizeof(dns));
    cfs_close(fd);
    //Get the real hostname
    int size = 0;
    //Because the space significate the end of the hostname
    while(dns[size] != '\0')
      ++size;
    //final = the real url
    char final[size];
    int i;
    for(i = 0; i != sizeof(final) +1; ++i)
      final[i] = dns[i];
    strcpy(coap_payload_beacon, "{\"n\":\"");
    strcat(coap_payload_beacon, final);
    strcat(coap_payload_beacon, "\"}");
  }
  else {
    printf("READING ERROR\n\r");
    strcpy(coap_payload_beacon, "{\"n\":\"default.test\"}");
    return;
  }
  printf("HOSTNAME : %s\n\r", coap_payload_beacon);
}
\end{minted}
Note: if the \emph{/HOSTNAME\_LORA} doesn't exists, the default name will be \emph{default.test}.
\end{document}

